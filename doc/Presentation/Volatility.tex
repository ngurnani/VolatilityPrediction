% Volatility Forecasting Project - MATH 287C Spring 2018
% Nishant D. Gurnani

%%%%%%%%%%%%%%%%%%%%%%%%%%%%%%%%%%%%%%%%%%%%%%%%%%%%%%%%%%%%%%%%%%%%%%%%%%%%%
\documentclass{beamer}
\mode<presentation> {
\usetheme{default}
\setbeamertemplate{footline}[page number]
\setbeamertemplate{section in toc}[sections numbered]
\setbeamercovered{invisible}
% To remove the navigation symbols from the bottom of slides%
\setbeamertemplate{navigation symbols}{} 
}

\usepackage{mathtools}
\DeclarePairedDelimiter{\ceil}{\lceil}{\rceil}


\newcommand{\id}{\mathrm{id}}
\newcommand{\K}{\mathrm{KL}}
\newcommand{\kl}{\mathrm{kl}}
\newcommand{\bp}{\boldsymbol{p}}
\renewcommand{\phi}{\varphi}
\renewcommand{\a}{\alpha}

\renewcommand{\P}{\mathbb{P}}
\newcommand{\E}{\mathbb{E}}
\newcommand{\N}{\mathbb{N}}
\newcommand{\R}{\mathbb{R}}
\newcommand{\Q}{\mathbb{Q}}
\newcommand{\KL}{\mathrm{KL}}
\newcommand{\LG}{\overline{\log}(d)}
\newcommand{\LGG}{\overline{\log}(M K)}
\newcommand{\ocP}{\overline{\mathcal{P}}}

\newcommand{\cO}{\mathcal{O}}
\newcommand{\cZ}{\mathcal{Z}}
\newcommand{\cA}{\mathcal{A}}
\newcommand{\cB}{\mathcal{B}}
\newcommand{\cN}{\mathcal{N}}
\newcommand{\cM}{\mathcal{M}}
\newcommand{\cF}{\mathcal{F}}
\newcommand{\cL}{\mathcal{L}}
\newcommand{\cX}{\mathcal{X}}
\newcommand{\cI}{\mathcal{I}}
\newcommand{\cJ}{\mathcal{J}}
\newcommand{\cY}{\mathcal{Y}}
\newcommand{\cH}{\mathcal{H}}
\newcommand{\cP}{\mathcal{P}}
\newcommand{\cT}{\mathcal{T}}
\newcommand{\cC}{\mathcal{C}}
\newcommand{\cS}{\mathcal{S}}
\newcommand{\cE}{\mathcal{E}}
\newcommand{\cK}{\mathcal{K}}
\newcommand{\cD}{\mathcal{D}}

\newcommand{\oD}{\overline{\mathcal{D}}}
\newcommand{\oR}{\overline{R}}

\def\ds1{\mathds{1}}
\renewcommand{\epsilon}{\varepsilon}

\newcommand{\wh}{\widehat}
\newcommand{\argmax}{\mathop{\mathrm{argmax}}}
\newcommand{\argmin}{\mathop{\mathrm{argmin}}}
\renewcommand{\mod}[2]{[#1 \,\, \mathrm{mod} \,\, #2]}
\newcommand{\todo}{{\bf TO DO } }

\renewcommand{\tilde}{\widetilde}

%Cadres d'algorithmes
\newlength{\minipagewidth}
\setlength{\minipagewidth}{\columnwidth}
\setlength{\fboxsep}{3mm}
\addtolength{\minipagewidth}{-\fboxrule}
\addtolength{\minipagewidth}{-\fboxrule}
\addtolength{\minipagewidth}{-\fboxsep}
\addtolength{\minipagewidth}{-\fboxsep}
\newcommand{\bookbox}[1]{\small
\par\medskip\noindent
\framebox[\columnwidth]{
\begin{minipage}{\minipagewidth} {#1} \end{minipage} } \par\medskip }

%%
%

\newcommand{\Ber}{\mathop{\mathrm{Ber}}}

\newcommand{\beq}{\begin{equation}}
\newcommand{\eeq}{\end{equation}}

\newcommand{\beqa}{\begin{eqnarray}}
\newcommand{\eeqa}{\end{eqnarray}}

\newcommand{\beqan}{\begin{eqnarray*}}
\newcommand{\eeqan}{\end{eqnarray*}}

\def\ba#1\ea{\begin{align*}#1\end{align*}} %\ba = \begin{algin*}, \ea = \end{align*}
\def\banum#1\eanum{\begin{align}#1\end{align}} %\banum = \begin{algin}, \eanum

%\newcommand{\qed}{\hfill\BlackBox}
\newcommand{\charfct}{\ds1} %
\newcommand{\Fcal}{\mathcal{F}}
\newcommand{\Xcal}{\mathcal{X}}
\newcommand{\Hcal}{\mathcal{H}}
\newcommand{\Gcal}{\mathcal{G}}
\newcommand{\Nat}{\mathbb{N}}


\newcounter{saveenumi}
\newcommand{\seti}{\setcounter{saveenumi}{\value{enumi}}}
\newcommand{\conti}{\setcounter{enumi}{\value{saveenumi}}}

\usepackage{natbib} % for citations
\usepackage{graphicx}
\usepackage{subfig}
\usepackage{bm} 

\newcounter{saveenumi}
\newcommand{\seti}{\setcounter{saveenumi}{\value{enumi}}}
\newcommand{\conti}{\setcounter{enumi}{\value{saveenumi}}}

\resetcounteronoverlays{saveenumi}

%\setbeamerfont{institute}{size=\medium}
\setbeamerfont{date}{size=\tiny}
%%%%%%%%%%%%%%%%%%%%%%%%%%%%%%%%%%%%%%%%%%%%%%%%%%%%%%%%%%%%%%%%%%%%%%%%%%%%%

\title[]{A Practical Look at Volatility in Financial Time Series}
\author{MATH 287C - Advanced Time Series Analysis \\ Nishant Gurnani}
\date{June 4th, 2018}

\begin{document}

% Title Slide
\begin{frame}
\titlepage
\end{frame}


% TOC SLIDE
\begin{frame}
\frametitle{Outline}
\tableofcontents[]
\end{frame}

% SECTION 1
\section{What is Volatility?}

\begin{frame}
\frametitle{Outline}
\tableofcontents[currentsection]
\end{frame}

\begin{frame}
\frametitle{What is Volatility?}
\begin{itemize}
\item Volatility is peculiar in that we know it exists, but we can't really measure it.
\item Conventional definitions rely on some distribution assumption on the returns, after which volatility is defined as the standard deviation $\sigma$ of the returns (to make things nicer we'll define it as )
\end{itemize}
\end{frame}

\begin{frame}
\frametitle{Naive Measure}
Include plot of rolling standard deviations on absolute returns
Describe why this is an unsatisfying measure
\end{frame}

\begin{frame}
\frametitle{Stylized Facts}
Describe stylized facts.
Show plot of S&P500 returns to reinforce these facts.
\end{frame}

\begin{frame}
\frametitle{GARCH}
Popular method to model volatility
Show formulas and how it's trained.
For the purposes of this talk we'll focus on GARCH(1,1) models.
\end{frame}

\section{Normalizing and Variance Stabilizing (NoVaS) Transformation}

\begin{frame}
\frametitle{Outline}
\tableofcontents[currentsection]
\end{frame}

\begin{frame}
\frametitle{NoVaS Transformation}
Include definition of NoVaS transformation and how it's derived.
Include algorithm for Simple NoVaS, which is the one we'll be focusing on for the purposes of this talk
\end{frame}


%% S&P 500 %%
\begin{frame}
\frametitle{S&P500}
pre-NoVas S&P500 returns plot
\end{frame}

\begin{frame}
\frametitle{S&P500}
pre-NoVas S&P500 returns histogram
\end{frame}

\begin{frame}
\frametitle{S&P500}
pre-NoVas S&P500 returns q-q plot
\end{frame}

\begin{frame}
\frametitle{S&P500}
post-NoVas S&P500 returns plot
\end{frame}

\begin{frame}
\frametitle{S&P500}
post-NoVas S&P500 returns histogram
\end{frame}

\begin{frame}
\frametitle{S&P500}
post-NoVas S&P500 returns q-q plot
\end{frame}

%% BTC %%
\begin{frame}
\frametitle{}
pre-NoVas BTC returns plot
\end{frame}

\begin{frame}
\frametitle{BTC}
pre-NoVas BTC returns histogram
\end{frame}

\begin{frame}
\frametitle{BTC}
pre-NoVas BTC returns q-q plot
\end{frame}

\begin{frame}
\frametitle{BTC}
post-NoVas BTC returns plot
\end{frame}

\begin{frame}
\frametitle{BTC}
post-NoVas BTC returns histogram
\end{frame}

\begin{frame}
\frametitle{BTC}
post-NoVas BTC returns q-q plot
\end{frame}

%% 5 min bar US Treasury Futures 10 year %%
\begin{frame}
\frametitle{Treasury Futures}
pre-NoVas Treasury Futures returns plot
\end{frame}

\begin{frame}
\frametitle{Treasury Futures}
pre-NoVas Treasury Futures returns histogram
\end{frame}

\begin{frame}
\frametitle{Treasury Futures}
pre-NoVas Treasury Futures returns q-q plot
\end{frame}

\begin{frame}
\frametitle{Treasury Futures}
post-NoVas Treasury Futures returns plot
\end{frame}

\begin{frame}
\frametitle{Treasury Futures}
post-NoVas Treasury Futures returns histogram
\end{frame}

\begin{frame}
\frametitle{Treasury Futures}
post-NoVas Treasury Futures returns q-q plot
\end{frame}

\begin{frame}
\frametitle{S&P500 not perfect tranform with NoVaS}
Simply show an imperfect transformation to make the point that financial time series over long periods are not necessary stationary (only locally stationary) and thus we should use time-varying versions of NoVaS where the window size isn't too big.
\end{frame}


\section{Forecasting Volatility}

\begin{frame}
\frametitle{Outline}
\tableofcontents[currentsection]
\end{frame}

\begin{frame}
\frametitle{One-Step Ahead Prediction}
Define the volatility prediction problem.
Outline that you're using squared returns $Y_{t}^2$ as a noisy proxy for Volatility
What loss function should you use?
Will use a window size of 250 days which is approx. 1 year
\end{frame}

\begin{frame}
\frametitle{Infinite Kurtosis?}
Do financial returns have infinite kurtosis?
If this is the case you, predicting under L2 is incorrect. Instead you should L1 loss where the median is optimal
\end{frame}

\begin{frame}
\frametitle{Infinite Kurtosis Plot S&P500}

\end{frame}

\begin{frame}
\frametitle{Infinite Kurtosis Plot BTC}

\end{frame}

\begin{frame}
\frametitle{Infinite Kurtosis Plot Treasury Futures}

\end{frame}

\begin{frame}
\frametitle{Prediction Intervals}
Steps for deriving prediction intervals - same for GARCH and NoVaS
\end{frame}

\begin{frame}
\frametitle{S&P500 Feb 2018 One Step Ahead Prediction}
Plot predicting S&P500 Feb 2018 Volatility spike, along with prediction intervals
Shows that Simple NoVaS is better than GARCH(1,1)
\end{frame}

\section{A Simple Volatility Trading Strategy}

\begin{frame}
\frametitle{Outline}
\tableofcontents[currentsection]
\end{frame}

\begin{frame}
\frametitle{Can predict $sigma^2$ using NoVaS under special conditions}
Talk about the conditions under which you can actually predict $sigma^2$, plot the ACF to confirm that transformed series is uncorrelated and independent.
\end{frame}

\begin{frame}
\frametitle{RV(t+1)-IV(t)}
Outline strategy that if RV(t+1)-IV(t) > 0 you buy VXX and vice versa.
\end{frame}

\begin{frame}
\frametitle{Strategy Results}
Cumulative returns plot, legend contains CAGR and Sharpe Ratio
\end{frame}

\end{document}